\subsection{Simulaciones con diferentes tipos de entrada}
\subsubsection{Seno}
Con la función seno, el carro se ve acelerado constantemente; pero en este caso, como
la función seno toma valores positivos y negativos, la aceleración se ejerce en
ambos sentidos lo que genera en el carro un movimiento o cambio en su posición
similar a la función de entrada.

\subsubsection{Coseno}
Con la función coseno, al igual que con seno, el carro se ve acelerando,
lo cual es esperado dado que la forma funcional de ambas entradas es similar;
sin embargo, al analizar las señales de salida estas son radicalmente diferentes.
Esto probablemente se deba a que la función empieza en su valor máximo y
disminuye progresivamente, indicando que el modelo funciona mejor si se
guía por funciones suaves.

\subsubsection{Escalón}
La función escalón muestra la variación del sistema ante la aplicación de la
fuerza. Puede observarse que antes del momento en que el escalón hace efecto,
el comportamiento es similar al de la función constante, es decir, la simulación
con fuerza externa sin variación o sin ésta. Después de la fuerza, que es permanente, puede verse que la
posición del vehículo cambia rápidamente debido a la aceleración permanente
que éste tiene.

\subsubsection{Sierra}
En la función sierra tiene un comportamiento bastante particular y al igual que
la función escalón, es fácil ver el lugar de la discontinuidad; sin embargo,
la discontinuidad de éste es mayor en cómo afecta la respuesta
temporal que probablemente se deba a la amplitud del salto.
