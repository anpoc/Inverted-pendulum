\documentclass{ieeeaccess}
\usepackage{cite}
\usepackage[utf8]{inputenc}
\usepackage{amsmath,amssymb,amsfonts}
\usepackage{algorithmic}
\usepackage{graphicx}
\usepackage{textcomp}
\def\BibTeX{{\rm B\kern-.05em{\sc i\kern-.025em b}\kern-.08em
    T\kern-.1667em\lower.7ex\hbox{E}\kern-.125emX}}
\begin{document}
\history{Date of publication xx 00, 0000, date of current version xxx 00, 0000.}
\doi{10.1109/ACCESS.2017.DOI}

\title{Inverted Pendulum}
\author{\uppercase{Andrea Posada Cárdenas\authorrefmark{1}},
  \uppercase{Santiago Hincapie Potes}\authorrefmark{2}}
\address[1]{Mathematical Engineering Student, Universidad EAFIT,
  Medellin, Colombia (e-mail: aposad31@eafit.edu.co)}
\address[2]{Mathematical Engineering Student, Universidad EAFIT,
  Medellin, Colombia (e-mail: shinca12@eafit.edu.co)}

\markboth
{Author \headeretal: Preparation of Papers for IEEE TRANSACTIONS and JOURNALS}
{Author \headeretal: Preparation of Papers for IEEE TRANSACTIONS and JOURNALS}

\corresp{Corresponding author: Andrea Posada Cárdenas
  (e-mail: aposad31@eafit.edu.co).}

\begin{abstract}
  TODO:Add abstract
\end{abstract}

\begin{keywords}
  TODO:Add keywords
\end{keywords}

\titlepgskip=-15pt

\maketitle

\section{Introduction}
\label{sec:introduction}
\PARstart{T}{he}

% Not sure the section
\subsection{Equation system}
\begin{align}
  (M + m)\ddot{x} + b\dot{x} + ml\theta\cos{\theta} - ml\dot{\theta}^2\sin{theta}
     = F \label{eq:NL1} \\
  (I + ml^2)\ddot{\theta} + mgl\sin{\theta} = -ml\ddot{x}\cos{\theta} \label{eq:NL2}
\end{align}
isolating the variable $\ddot{x}$ in~\ref{eq:NL1} we get,
\[
  \ddot{x} = \frac{F - b\dot{x} - ml\ddot{\theta}\cos{\theta} + ml\dot{\theta}^2\sin{\theta}}{M + m}
\]
substituting in~\ref{eq:NL2} and isolating $\ddot{\theta}$ we get
\begin{equation}
  \label{eq:NLddth}
  \ddot{\theta} = \frac{-ml(F - b\dot{x} + ml\dot{\theta}^2\sin{\theta}) - mgl(M + m)\sin{\theta}}
       {I(M + m) + Mml^2 + m^2l^2\sin^2{\theta}}
\end{equation}
Similarly, isolating $\ddot{\theta}$ in~\ref{eq:NL2} we get,
\[
  \ddot{\theta} = \frac{-ml(\ddot{x}\cos{\theta} + g\sin{\theta})}{I + ml^2}
\]
substituting and isolating $\ddot{x}$, we get
\begin{equation}
  \label{eq:NLddx}
  \ddot{x} = \frac{(I + ml^2)(F - b\dot{x} + ml\dot{\theta}^2\sin{\theta})
    + m^2gl^2\sin{\theta}\cos{\theta}}{I(M + m) + Mml^2 + m^2l^2\sin^2{\theta}}
\end{equation}

\subsection{non-linear state space model}
\subsubsection{state space variables}
\begin{align*}
  x_1 &= x       & \text{cart position}    \\
  x_2 &= \theta       & \text{pendulum angle}   \\
  x_3 &= \dot{x} & \text{cart velocity}    \\
  x_4 &= \dot{\theta} & \text{angular velocity} \\
  u   &= F       & \text{input force}
\end{align*}
\subsection{state space equation system}
\begin{equation}
  \label{eq:NLM}
  \begin{aligned}
    \dot{x_1} &= x_3 \\
    \dot{x_2} &= x_4 \\
    \dot{x_3} &= \frac{(I + ml^2)(F - bx_3 + mlx_4^2\sin{x_2})
      + m^2gl^2\sin{x_2}\cos{x_2}}{I(M + m) + Mml^2 + m^2l^2\sin^2{x_2}} \\
    \dot{x_4} &= \frac{-ml(F - bx_3 + mlx_4^2\sin{x_2}) - mgl(M + m)\sin{x_2}}
       {I(M + m) + Mml^2 + m^2l^2\sin^2{x_2}}
  \end{aligned}
\end{equation}

\subsection{linearization}
When $x_2 \to 0$,
\begin{itemize}
  \item $\sin{x_2} \to x_2$
  \item $\cos{x_2} \to 1$
  \item $\sin^2{x_2} \to x_2^2 \approx 0$
  \item $x_4^2 \approx 0$
\end{itemize}
therefore, we can express~\ref{eq:NLM} when $\theta \to 0$ as,
\begin{equation}
  \label{eq:LME}
  \begin{aligned}
    \dot{x_1} &= x_3 \\
    \dot{x_2} &= x_4 \\
    \dot{x_3} &= \frac{(I + ml^2)(F - bx_3) + m^2gl^2x_2}{I(M + m) + Mml^2} \\
    \dot{x_4} &= \frac{-ml(F - bx_3) - mgl(M + m)x_2}{I(M + m)}
  \end{aligned}
\end{equation}
or in matrix form
\begin{equation}
  \label{eq:LM}
  \begin{aligned}
    \dot{x} &= Ax + Bu \\
    y       &= Cx + Du
  \end{aligned}
\end{equation}
where
\begin{align*}
  x &= \begin{pmatrix} x_1 \\ x_2 \\ x_3 \\ x_4 \end{pmatrix} \in \mathbb{R}^{4 \times 1} \\
  u &= F \in \mathbb{R} \\
  y &= \begin{pmatrix} y_1 \\ y_2 \end{pmatrix} \in \mathbb{R}^{2 \times 1} \\
  A &= \frac{1}{I(M + m) + Mml^2} \begin{pmatrix}
    0 & 0 & 1 & 0 \\
    0 & 0 & 0 & 1 \\
    0 & m^2gl^2 & -(I + ml^2)b & 0 \\
    0 & -mgl(M + m) & mlb & 0
  \end{pmatrix} \in \mathbb{R}^{4 \times 4} \\
  B &= \frac{1}{I(M + m) + Mml^2} \begin{pmatrix}
    0 \\ 0 \\ I + ml^2 \\ -ml
  \end{pmatrix} \in \mathbb{R}^{4 \times 1} \\
  C &= \text{Andrea!!! we need to define our outputs}  \in \mathbb{R}^{4 \times 2} \\
  D &= 0 \in \mathbb{R}^{2 \times 1}
\end{align*}
\EOD{}
\end{document}
