% analice la estabilidad de la planta para dos
%  parámetros diferentes. Posteriormente, aplique un control con un regulador estático (u = k
%  x error) y determine los valores para los cuales el sistema es estable e inestable. Analice los
%  resultados.

\begin{eqnarray}
  \label{eq:sistema}
  \left\{
  \begin{array}{ll}
    f_1 = \displaystyle\dot{x_1} = \displaystyle x_2 \\
    f_2 = \displaystyle\dot{x_2} =
            \displaystyle\frac{(I + ml^2)a + m^2l^2g\sin(x_3)\cos(x_3)}
              {(M + m)(I + ml^2)-m^2l^2\cos^2(x_3)}\\
    f_3 = \displaystyle\dot{x_3} = \displaystyle x_4\\
    f_4 = \dot{x_4} = \displaystyle\frac{-ml\cos(x_3)a - (M + m)mlg\sin(x_3)}
                         {(M + m)(I + ml^2) - m^2l^2\cos^2(x_3)}\\
  \end{array}
  \right.
\end{eqnarray}

\subsection{Método de Routh-Hurwitz}
Análisis de estabilidad lineal en tiempo continuo en el punto de operación seleccionado. El polinomio característico está dado por:
\begin{equation}
  \label{eq:polcar}
  p(s) = s^3 + 0.1818 s^2 + 31.2030 s + 4.4576 \\
\end{equation}
